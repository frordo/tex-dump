\documentclass[11]{article}
\usepackage{amsmath}
\usepackage{csquotes}
\usepackage{amssymb}
\usepackage{hyperref}
\newlength\tindent
\setlength{\tindent}{\parindent}
\setlength{\parindent}{0pt}
\renewcommand{\indent}{\hspace*{\tindent}}

\title{\sc{Proof of Cauchy-Schwarz by Brute Force}}
\author{\sc{Rahul}}
\date{13 March, 2025}
\begin{document}
\maketitle

Suppose that $f$ and $f'$ are continuous on $\mathbb{R}$, and that $\displaystyle\lim_{x\to\infty}f(x)$ and $\displaystyle\lim_{x\to\infty}f'(x)$ both exist. Then show that $$\displaystyle\lim_{x\to\infty}f'(x) = 0$$


This is quite intuitive. If $f$ converges to a fixed number, then $f'$ must converge to zero (otherwise $f$ would keep changing and not converge to a fixed number). In other words, it is a horizontal asymptote. How do we formalize this intuition?

The answer, is of course the mean value theorem! (it is always the mean value theorem :D)

There exists a $c_n \in [n, n+1]$ such that
$$f(n+1) - f(n) = f'(c_n)$$
$$\lim_{n\to\infty} f(n+1) - f(n)= \lim_{x\to\infty} f'(c_n)$$

$$\lim_{c_n\to\infty} f'(c_n) = \lim_{n\to\infty} f(n+1) - f(n) = 0$$

Since the limit is given to exist, and it equals to zero for one sequence, it must be equal to zero for all sequences.

If you relax the conditions of the problem, say for example if you do not required that $\displaystyle\lim_{x\to\infty}f'(x)$ exists, the proof will not work. For example consider $f(x) = \frac{\sin x^2}{x}$ (credit to user \href{https://math.stackexchange.com/questions/42277/proving-that-lim-limits-x-to-inftyfx-0-when-lim-limits-x-to-inftyf/42298#comment5425701_42277}{Daniel on math.SE}).

Another way (nearly the same as the MVT one):
$$\lim_{x\to\infty}\frac{f(x)}{x} = \lim_{x\to\infty}f'(x) = 0$$

Here is a very slick proof: (using L'Hopital)
$$\lim_{x\to\infty}f(x) = \lim_{x\to\infty}\frac{f(x)e^x}{e^x} = \lim_{x\to\infty}\frac{e^x(f(x) + f'(x)}{e^x} = \lim_{x\to\infty} f(x) + f'(x)$$
$$\lim_{x\to\infty} f'(x) = 0$$
\end{document}