\documentclass[11pt,a4paper]{article}
\usepackage[utf8]{inputenc}
\usepackage{amsmath}
\usepackage{amsfonts}
\usepackage{amssymb}

\title{\sc{Notes on Euler's Theorem}}
\author{\sc{Rahul}}
\date{\today}
\begin{document}
\maketitle

\section{Introduction}

Currently unfinished -- to do
Which one of Euler's many theorems do I refer to? It is the one on homogeneous functions. To start, let us define:

\subsection{Homogeneous Functions}
A function $f: \mathbb{R}^k \to \mathbb{R}$ is said to be homogeneous of degree $n$ if\[f(tx_1, tx_2, \dots, tx_k) = t^n f(x_1, x_2, \dots, x_k) \]or using vector notation
\[f(t \mathbf{x}) = t^n f(\mathbf{x})\]

For instance, in a \emph{homo}geneous polynomial, all terms have the same degree. \underline{Example}: $f(x, y) = x^2 + y^2$. Note that $n$ need not be a positive integer. One may also be familiar with the term homogeneous differential equations.

\section{Euler's Theorem}
\subsection{Statement}
If $f: \mathbb{R}^k \to \mathbb{R}$ is continuously differentiable and homogeneous of degree $n$, then
\[\sum_{i=1}^k x_i \cdot \frac{\partial f}{\partial x_i} = n f(x_1, x_2, \dots, x_k) \]or in vector notation\[\nabla f \cdot \mathbf{x} = n f(\mathbf{x})\]
\subsubsection{Corollary 1}
For a single variable, \[x \frac{\partial f}{\partial x} = x \frac{df}{dx} = nf \iff f(x) = cx^n\]
\subsubsection{Corollary 2}
For two variables,\[n f(x, y) = x f_x + y f_y\]
\subsection{Proof of Euler's Theorem}
\[f(tx_1, tx_2, \dots, tx_k) = t^n f(x_1, x_2, \dots, x_k)\label{eq:1} \tag{1}\]
Differentiating the LHS of $\eqref{eq:1}$ with respect to $t$, and applying the chain rule,\[\frac{df(t\mathbf{x})}{dt} = \sum_{i=1}^k\frac{\partial f(tx_i)}{\partial tx_i} \frac{d(tx_i)}{dt} = \frac1t \sum_{i=1}^k\frac{\partial f(tx_i)}{\partial x_i} x_i \label{eq:2} \tag{2}\]Differentiating the RHS of $\eqref{eq:1}$ with respect to $t$, \[nt^{n-1} f(\mathbf{x}) = \frac{n t^n f(\mathbf{x})}{t} = \frac{n}{t} f(\mathbf{tx}) \label{eq:3} \tag{3}\]Equating $\eqref{eq:2}$ and $\eqref{eq:3}$, and setting $t = 1$ gives the desired result. \begin{flushright}$\square$\end{flushright}

\subsection{Converse of Euler's Theorem}
a
\subsection{Interesting Result}
\[xf_x + yf_y = nf \label{eq:a} \tag{1}\]
\[ \left(\frac{\partial}{\partial x} \ (1) \implies x f_{xx} + f_x + yf_{yx} = n f_x \right) \label{eq:b} \tag{2.1}\]
\[\left(\frac{\partial}{\partial y} \ (1) \implies y f_{yy} + f_y + x f_{xy} = n f_y \right)\label{eq:c} \tag{2.2}\]
$(2.1) \times x + (2.2) \times y$
\[(x^2 f_{xx}+y^2 f_{yy}) + (xf_x + yf_y) + xy (f_{xy} + f_{xy}) = n (xf_x + yf_y) \]
Let $f$ be sufficiently nice--let the second derivative of $f$ be continuous. 
\[(x^2 f_{xx}+y^2 f_{yy}) + 2xyf_{xy} = n(n-1)f \]
Where we have used $(1)$ and $f_{xy} = f_{yx}$.
Denote $f_{\underbrace{xxx}_\text{$n$ times}} = f_{x^n}$\\
We claim that this generalizes. For the case ${k=3}$,
\[x^3 f_{x^3} + y^3 f_{y^3} + 3x^2 y f_{x^2y} + 3y^2x f_{y^2x} = n(n-1)(n-2)f\]

In general,

\[ \sum_{r=0}^k \binom{k}{r} x^r y^{k-r} f_{x^r y^{k-r}} = n(n-1)\dotsm(n-(k-1))f \]

\subsubsection*{Proof}
The base case $2$ is true. Suppose the case $k$ is true. We have to show that the case $k+1$ is true.

\[ \sum_{r=0}^k \binom{k}{r} x^r y^{k-r} f_{x^r y^{k-r}} = n(n-1)\dotsm(n-(k-1))f \]

\[ \sum_{r=0}^k \binom{k}{r} r x^{r-1} y^{k-r} f_{x^r y^{k-r}} + \sum_{r=0}^k \binom{k}{r}  x^{r} y^{k-r} f_{x^{r+1} y^{k-r}} = n(n-1)\dotsm(n-(k-1))f_x \]

\[ \sum_{r=0}^k \binom{k}{r} x^r (k-r)y^{k-r-1} f_{x^r y^{k-r}} + \sum_{r=0}^k \binom{k}{r} x^r y^{k-r} f_{x^r y^{k-r+1}} = n(n-1)\dotsm(n-(k-1))f_y \]

\[\sum_{r=0}^k \binom{k}{r} r x^{r} y^{k-r} f_{x^r y^{k-r}} + \sum_{r=0}^k \binom{k}{r} x^r (k-r)y^{k-r} f_{x^r y^{k-r}} = k \sum_{r=0}^k \binom{k}{r} x^r y^{k-r} f_{x^r y^{k-r}}\] 

\[ \sum_{r=0}^k \binom{k}{r}  x^{r+1} y^{k-r} f_{x^{r+1} y^{k-r}} + \sum_{r=0}^k \binom{k}{r} x^r y^{k-r+1} f_{x^r y^{k-r+1}}\]

\end{document}